\section{Conclusion}
\label{sec:conclusion}

This survey synthesized embodied AI and world-model research from 2024 to early 2026 under a coupled framework that links system-level embodied decision stacks with model-level dynamics design choices. The central conclusion is that strong embodied performance now depends on explicit coordination among representation, prediction, and control, rather than progress in any single module.

Beyond conceptual synthesis, we provide an auditable citation workflow over 318 collected papers, with 279 in-scope entries under explicit time-window and embodied-closure criteria, plus grouped inclusion and exclusion appendices. This structure is intended to make future updates incremental and reproducible as the field continues to evolve rapidly.

Our overall reading of the current frontier is pragmatic: foundation-scale pretraining has become necessary but not sufficient. Reliable embodied intelligence increasingly requires decision-coupled post-training, representation interfaces aligned with embodiment constraints, and deployment-oriented evaluation protocols that quantify not just task completion, but sustained autonomy quality.
