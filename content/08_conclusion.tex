\section{Conclusion}
\label{sec:conclusion}

This survey synthesized embodied AI and world-model research from 2024 to early 2026 under a coupled framework that links system-level embodied decision stacks with model-level dynamics design choices. The central conclusion is that strong embodied performance now depends on explicit coordination among representation, prediction, and control, rather than progress in any single module.

The technical trajectory is cumulative: pre-2024 advances in task definition, language grounding, and early generalist robot policies established the interfaces that 2024--2026 systems now optimize at scale. Recent progress therefore looks less like a paradigm replacement and more like integration of planning, world modeling, and policy adaptation into a single closed-loop training and deployment stack.

Our overall reading of the current frontier is pragmatic: foundation-scale pretraining has become necessary but not sufficient. Reliable embodied intelligence increasingly requires decision-coupled post-training, representation interfaces aligned with embodiment constraints, and deployment-oriented evaluation protocols that quantify not just task completion, but sustained autonomy quality.
