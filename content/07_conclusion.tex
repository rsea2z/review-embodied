\section{Conclusion and Future Outlook}
\label{sec:conclusion}

This survey has systematically explored the transformative landscape of World Models for Embodied AI during the pivotal 2025--2026 period. We have traced the field's rapid evolution from specialized, model-free control policies toward generalist architectures grounded in high-fidelity predictive dynamics and fueled by massive, heterogeneous data engines. Our analysis reveals that the core innovation defining this era is \textbf{Physicalization}---the transition from "dreamy" video generation to precise, action-conditional world modeling. By internalizing the laws of physics and the causal structure of interaction, world models have graduated from auxiliary components to the foundational pillars of embodied intelligence, enabling agents to reason about temporal transitions, anticipate the consequences of their interventions, and generalize across the reality gap with unprecedented robustness.

The convergence of three critical trends has enabled this progress. First, the emergence of unified \textbf{Data Engines} has solved the historical data bottleneck by synthesizing internet-scale video priors with high-intensity robot interaction data, as exemplified by initiatives like Open X-Embodiment \cite{collaboration_open_2025} and NVIDIA Cosmos \cite{wang_genie_2025}. Second, the shift toward \textbf{Flow Matching and Latent-Space Predictive Architectures} has provided the mathematical maturity required for stable, high-frequency control, overcoming the limitations of stochastic diffusion and discrete autoregression. Third, the transformation of \textbf{Simulation into an Active Data Generator}---through procedural "Infinite Worlds" and photorealistic digital twins---has allowed models to learn from a distribution of experiences far broader than what is physically attainable, effectively addressing the "long-tail" scenarios essential for safe real-world deployment.

Looking ahead, we anticipate the dissolution of traditional boundaries between perception, prediction, and control. The trajectory points toward a \textbf{Unified World Model}---a single, massive neural architecture that consumes multimodal history and outputs both future rollouts and control actions within a shared, actionable latent space. This next generation of "System-2" embodied agents will likely integrate the deliberative "Chain-of-Thought" reasoning of LLMs directly into the physics prediction loop, enabling strategic planning over long horizons. As we move closer to closing the Sim-to-Real loop through Real-to-Sim-to-Real experience replay, world models will serve as the "physical conscience" of autonomous systems, paving the way for the first generation of truly capable, generalist agents that can navigate, manipulate, and master the complexity of the physical world.
